%% LyX 1.5.6 created this file.  For more info, see http://www.lyx.org/.
%% Do not edit unless you really know what you are doing.
\documentclass[12pt,british]{article}
\batchmode
\usepackage{times}
\usepackage{helvet}
\usepackage{courier}
\usepackage[T1]{fontenc}
\usepackage[latin9]{inputenc}
\usepackage{geometry}
\geometry{verbose,letterpaper,tmargin=1in,bmargin=1in,lmargin=1in,rmargin=1in}
\pagestyle{empty}
\usepackage{array}
\usepackage{textcomp}
\usepackage{url}
\usepackage{longtable}

\makeatletter

%%%%%%%%%%%%%%%%%%%%%%%%%%%%%% LyX specific LaTeX commands.
\providecommand{\LyX}{L\kern-.1667em\lower.25em\hbox{Y}\kern-.125emX\@}
\newcommand{\noun}[1]{\textsc{#1}}
%% Because html converters don't know tabularnewline
\providecommand{\tabularnewline}{\\}

%%%%%%%%%%%%%%%%%%%%%%%%%%%%%% Textclass specific LaTeX commands.
\newenvironment{lyxlist}[1]
{\begin{list}{}
{\settowidth{\labelwidth}{#1}
 \setlength{\leftmargin}{\labelwidth}
 \addtolength{\leftmargin}{\labelsep}
 \renewcommand{\makelabel}[1]{##1\hfil}}}
{\end{list}}

%%%%%%%%%%%%%%%%%%%%%%%%%%%%%% User specified LaTeX commands.
% \renewcommand{\rmdefault}{sfdefault}
\usepackage{pslatex}

\usepackage{rcs}
\RCS $Date: 2011/04/05 23:19:03 $
\RCS $Revision: 1.34 $

\makeatother

\usepackage{babel}

\begin{document}
\noindent{} Version \RCSRevision{}; \RCSDate{}


\section*{\LaTeX\  and \LyX{} Tips }

\texttt{}\begin{tabular}{l>{\raggedright}p{4in}}
\multicolumn{2}{l}{\textbf{Administrative}}\tabularnewline
\texttt{texhash} & update \LaTeX\ database\tabularnewline
 & \tabularnewline
\multicolumn{2}{l}{\textbf{\LyX{} Shortcuts}}\tabularnewline
\texttt{{[}alt-C]{[}space] } & default character \tabularnewline
\texttt{{[}alt-P]{*}2} & unnumbered section \tabularnewline
\texttt{{[}alt-P]2} & numbered section\tabularnewline
\texttt{{[}alt-P]i } & item\tabularnewline
\texttt{{[}alt-P]l} & list\tabularnewline
\texttt{{[}alt-P]e} & enumerate\tabularnewline
\texttt{{[}alt-C]{[}right]} & Capitalize\tabularnewline
\texttt{{[}alt-C]{[}down]} & capitalize\tabularnewline
\texttt{{[}alt-C]{[}up]} & CAPITALIZE\tabularnewline
\texttt{{[}alt-C]p} & typewriter font\tabularnewline
\texttt{{[}alt-S]5} & font-size normal, increase numbers for larger font\tabularnewline
\texttt{{[}alt-S]n} & font-size normal\tabularnewline
\texttt{{[}F7]} & spellchecker\tabularnewline
 & \tabularnewline
\multicolumn{2}{l}{\textbf{\LaTeX\ Commands}}\tabularnewline
\texttt{\textbackslash{}discretionary\{\%1\}\{\%2\}\{\%3\}} & \parbox[t]{4in}{``Hyphenation'' at line breaks, before\%3after, before\%1\\
\%2after\\
}\tabularnewline
\texttt{\textbackslash{}slash\{\}} & Breakable slash, short cut for  \texttt{\textbackslash{}discretionary\{/\}\{\}\{/\}}\tabularnewline
\texttt{\textbackslash{}filbreak} & prevent an item in a (for eg.) bibliography list breaking across page
breaks\tabularnewline
\texttt{\textbackslash{}jobname} & the current filename \tabularnewline
\texttt{\textbackslash{}filename@ext} & the current filename \tabularnewline
 & \tabularnewline
\texttt{\textbackslash{}ldots } & ellipsis \ldots \tabularnewline
\verb, -- , & en-dash -- \tabularnewline
\verb, --- , & em-dash --- \tabularnewline
 & \tabularnewline
\multicolumn{2}{l}{In Math Mode (in \texttt{\$math equation\$}):}\tabularnewline
\texttt{\textbackslash{}; } & a thick space\tabularnewline
\texttt{\textbackslash{}: } & a medium space\tabularnewline
\texttt{\textbackslash{}, } & a thin space\tabularnewline
\texttt{\textbackslash{}! } & a negative thin space\tabularnewline
\texttt{\textbackslash{}times} & $\times$\tabularnewline
 & \tabularnewline
\end{tabular}

\begin{flushleft}
\begin{tabular}{ll}
\textbf{\textsc{Bib}\TeX{}} & \tabularnewline
\texttt{\textbackslash{}usepackage{[}square,sort]\{natbib\} } & in preamble\tabularnewline
\texttt{\textbackslash{}renewcommand\textbackslash{}refname\{References\}} & in preamble to change heading\tabularnewline
\texttt{\textbackslash{}bibliographystyle\{ametsoc.bst\}}  & declare the \textsc{Bib}\TeX{} style file to use\tabularnewline
\texttt{\textbackslash{}bibliography\{abs,references\}\{\} } & in document where the bibliography is to appear\tabularnewline
\end{tabular}
\par\end{flushleft}

\newpage


\section*{\texttt{xdvi} Commands}

\begin{tabular}{ll}
\texttt{R} & refresh\tabularnewline
\textbf{n}\texttt{s} & scale\tabularnewline
\texttt{g}, \textbf{n}\texttt{g} & end, move to page \textbf{n}\tabularnewline
\texttt{n}, \textbf{n}\texttt{n} & next page, \textbf{n}th next page\tabularnewline
\texttt{p}, \textbf{n}\texttt{p} & previous page, \textbf{n}th prev. page\tabularnewline
\texttt{k} & toggle keep position\tabularnewline
\texttt{D} & toggle grid\tabularnewline
\end{tabular}


\section*{\texttt{ssh} Commands}

\begin{tabular}{ll}
\texttt{\textasciitilde{}?} & help\tabularnewline
\texttt{\textasciitilde{}\#} & display\tabularnewline
\end{tabular}

\noindent{}\begin{tabular}{cc}
\texttt{ssh -X} & force X connection\tabularnewline
\end{tabular}


\section*{\texttt{vi} Commands}

\begin{tabular}{ll}
\texttt{:com <name> {[}-range] {[}-nargs]} & create macro command\tabularnewline
\texttt{|} & separates vim commands on same line\tabularnewline
\texttt{:sp} & split window\tabularnewline
\texttt{{[}\textasciicircum{}w]{[}down]}, \texttt{{[}\textasciicircum{}w]{[}up]},
\texttt{{[}\textasciicircum{}w]w} & move around windows\tabularnewline
\texttt{:\char`\"{}} & comment\tabularnewline
\texttt{{[}\textasciicircum{}f}], \texttt{{[}\textasciicircum{}b]} & forward, back screen\tabularnewline
\texttt{:g/\textasciicircum{}/norm J} & join every second line \tabularnewline
 & \tabularnewline
\end{tabular}


\subsubsection*{vim split windows commands}

\noindent \texttt{}\begin{tabular}{|l|l|}
\hline 
\noindent \texttt{\textasciicircum{}w\textasciicircum{}w } & \noindent change to next \tabularnewline
\hline 
\noindent \texttt{\textasciicircum{}w\textasciicircum{}n } & \noindent new empty window \tabularnewline
\hline 
\noindent \texttt{\textasciicircum{}wq } & \noindent quit window \tabularnewline
\hline 
\noindent \texttt{\textasciicircum{}wc } & \noindent close window \tabularnewline
\hline 
\noindent \texttt{\textasciicircum{}wx } & \noindent exchange windows\tabularnewline
\hline 
\noindent \texttt{\textasciicircum{}w= } & \noindent equalize \tabularnewline
\hline 
\noindent \texttt{\textasciicircum{}w+ } & \noindent increase \tabularnewline
\hline 
\noindent \texttt{\textasciicircum{}w- } & \noindent decrease \tabularnewline
\hline 
\noindent \texttt{\textasciicircum{}w\_ } & \noindent maximise \tabularnewline
\hline 
\noindent \texttt{\textasciicircum{}wo} & \noindent  make only window\tabularnewline
\hline
\end{tabular}

\newpage
\section*{\texttt{bash} Scripting and \noun{Unix} Commands}


% \begin{tabular}{>{\raggedright}p{2in}p{4in}}
\begin{longtable}{>{\raggedright}p{2in}p{4in}}
\texttt{man} command \texttt{| col -b } & Print plain \noun{ascii} \texttt{man} page (\texttt{col} filters \texttt{ANSI}
escapes)\tabularnewline
\texttt{grep -B3 -C3 } & Find with context before and after\tabularnewline
\texttt{mktemp} & Create unique temporary file\tabularnewline
 & \tabularnewline
\multicolumn{2}{l}{\textbf{Customize Keyboard}}\tabularnewline
\texttt{dumpkeys > newkeyfile} & Dump current keymap to a file \tabularnewline
\texttt{loadkeys newkeyfile} & Load new keymap from a file \tabularnewline
\texttt{keycode 29 = Caps\_Lock}\\ \texttt{keycode 58 = Control} &
	Swap capslock and control \tabularnewline
 & \tabularnewline
\multicolumn{2}{l}{To allow <ctrl-alt-end> to shutdown}\tabularnewline
\multicolumn{2}{l}{\texttt{control alt keycode 79 = KeyboardSignal}} \\                            
\multicolumn{2}{l}{\texttt{control alt keycode 107 = KeyboardSignal}} \\
\multicolumn{2}{l}{and add \texttt{kb::kbrequest:/sbin/shutdown -h now} to  inittab} 
 \tabularnewline
 & \tabularnewline
\texttt{showkey} &	to display the codes of the key pressed \tabularnewline
\texttt{xkeycap} &	to set up modmap graphically \tabularnewline

NB Read the keyboard and console howto!  & \tabularnewline
 & \tabularnewline
\multicolumn{2}{l}{\textbf{Command-Line Keystrokes}}\tabularnewline
\texttt{\textasciicircum{}u} & delete line\tabularnewline
\texttt{\textasciicircum{}c} & cancel command\tabularnewline
\texttt{\textasciicircum{}d}, \texttt{<esc>-d} & delete character, word\tabularnewline
\texttt{\textasciicircum{}w} & delete word backwards\tabularnewline
\texttt{\textasciicircum{}v} & quote character\tabularnewline
\texttt{\textasciicircum{}t}, \texttt{alt-t} & transpose chars, words\tabularnewline
\texttt{\textasciicircum{}f}, \texttt{alt-f} & forward\tabularnewline
\texttt{\textasciicircum{}a} & start of line\tabularnewline
\texttt{\textasciicircum{}e} & end of line\tabularnewline
 & \tabularnewline
\multicolumn{2}{l}{\textbf{History Substitution}}\tabularnewline
\texttt{!!} & repeat previous command \tabularnewline
\texttt{!n:k} & insert the \verb=k=-th word of command \verb=n= \tabularnewline
\texttt{!-n:k} & insert the \verb=k=-th word of \verb=n=-th previous command \tabularnewline
\texttt{!-n:*} & insert all the words but the \verb=0=-th of \verb=n=-th previous command \tabularnewline
 & \tabularnewline 
%
\textbf{System Administration} & \tabularnewline
\texttt{service}, \texttt{/etc/init.d/{*}d} & start, stop and status of daemons\tabularnewline
\texttt{chkconfig} & manipulate run levels\tabularnewline
\texttt{hwclock -r}, \texttt{-a}, \texttt{-w}, \texttt{-s} & read, adjust, write to, set from the hardware clock\tabularnewline
\texttt{procmail}, \texttt{fetchmail}, \texttt{sendmail} & mail processing\tabularnewline
\texttt{procinfo}, \texttt{top}, \texttt{uptime}, \texttt{w}, \texttt{who},
\texttt{whoami} & information about the system\tabularnewline
\texttt{lsof}, \texttt{fuser} & information about open processes\tabularnewline
\texttt{dmidecode} & hardware information \tabularnewline
\texttt{siga} & System Information GAtherer --- SuSE system info
tool \tabularnewline
\multicolumn{2}{l}{See \url{http://www.cpqlinux.com/hostname.html} about fixing the hostname}
\tabularnewline

 & \tabularnewline
%
\pagebreak[4]\textbf{Monitoring} & \tabularnewline
\texttt{netwatch}, \texttt{iptraf}, \texttt{iftop} & \tabularnewline
\texttt{mii-tool} & interface information\tabularnewline
\texttt{ntop:3000} & \tabularnewline
 & \tabularnewline

\pagebreak[3]\multicolumn{2}{l}{\textbf{Turning off the annoying beep}}\tabularnewline
\texttt{set bell-style none} & \parbox[t]{4in}{in \texttt{\textasciitilde{}/.inputrc,} \texttt{\textasciitilde{}/.profile,}
or \texttt{\textasciitilde{}/.bash-profile}.\\
Does not work in \texttt{.bashrc}}\tabularnewline
\texttt{set nobeep=1} & in \texttt{csh}\tabularnewline
\texttt{xset b off} & in X-window\tabularnewline
\texttt{{}``Settings$\rightarrow$Bell$\rightarrow$None''} &  in \texttt{konsole} \tabularnewline
 & \tabularnewline
\textbf{Other useful stuff} & \tabularnewline
\texttt{x-friend}, \texttt{google-desktop} & desktop search\tabularnewline
 & \tabularnewline
 & \tabularnewline
% \end{tabular}
\end{longtable}

Some thoughts to add: to render a man page to plain ascii use \texttt{man man |
col -b}

\newpage


\section*{GMT Hints}

\texttt{pstext} input: (\emph{x}, \emph{y}, \emph{size}, \emph{angle},
\emph{fontno}, \emph{justify}, \emph{text})

\noindent{}\texttt{convert -density 150 -page A4 filename.ps filename.png}

\noindent{}\texttt{gmtset WANT\_EURO\_FONT true} to get europeaen
character sets

\begin{tabular}{cc}
\multicolumn{2}{c}{}\tabularnewline
\multicolumn{2}{c}{Character Table}\tabularnewline
\multicolumn{2}{c}{}\tabularnewline
\textdegree{}  & \textbackslash{}217\tabularnewline
$\sigma$ & \textbackslash{}163\tabularnewline
$\Theta$ & \textbackslash{}161\tabularnewline
� & \textbackslash{}370\tabularnewline
$\Delta$ & \textbackslash{}104\tabularnewline
 & \tabularnewline
\end{tabular}

\begin{tabular}{cc}
\texttt{@\textasciitilde{}} & to toggle symbol font\tabularnewline
 & \tabularnewline
\end{tabular}


\section*{\texttt{wget} Options}

\begin{tabular}{ll}
\texttt{-p} & everything needed\tabularnewline
\texttt{-nH} & not under host directory\tabularnewline
\texttt{-{}-cut-dirs=n} & ignore leading directory tree\tabularnewline
\texttt{-r } & recursive\tabularnewline
\texttt{-N} & timestamping\tabularnewline
\texttt{-np} & no parent\tabularnewline
\texttt{-nv} & nonverbose\tabularnewline
\texttt{-Q} & quota\tabularnewline
 & \tabularnewline
 & \tabularnewline
\end{tabular}


\section*{\texttt{rsync} Options}

\texttt{rsync {[}options] fromdir host:destdir}

\begin{tabular}{ll}
 & \tabularnewline
\texttt{-r} & recursive\tabularnewline
\texttt{-t} & copy timestamps too\tabularnewline
\texttt{-u } & update newer only\tabularnewline
\texttt{-n} & test, don't do it\tabularnewline
\texttt{-v} & verbose\tabularnewline
\end{tabular}

\clearpage


\section*{Manipulating \texttt{PostScript} Documents}


\begin{enumerate}
\item StarOffice and OpenOffice
\begin{enumerate} % {{{
\item Creating a presentation from StarOffice

\begin{enumerate} % {{{
\item print as a .ps, using the trim option
\item Rotate using \texttt{}~\\
\texttt{pstops -w0 -h0 1:0R\textbackslash{}(0in,8.27in\textbackslash{})
psfile > rotfile}
\item \texttt{pstopdf -g7930x5950 rotfile pdffile}
\end{enumerate} % }}}
Now it's in a script \texttt{rotate.zsh}

\item Preparing figures for OpenOffice

	OpenOffice is very bad at eps figures. Turn them into JPEGs. Matlab
JPEGs are very bad, print them as EPS and turn them into JPEGs with
\texttt{gimp}. \texttt{gmt} does not make JPEGs, make eps figures and use
\texttt{gimp}. 


\end{enumerate} % }}}

\item Changing from EPS to PS


Use epsffit \\
A4 595x842

A5 421x595

A6 297x421

A7 210x297

Using pstops

\texttt{pstops '2:0L@.65(21cm,0)+1L@.65(21cm,14.85)' filename}

\item Converting to Postscript


\texttt{convert -density {[}density] fromfile.jpg tofile.ps }

\texttt{density} here refers to number of pixels across? (see ImageMagick
help pages)

\texttt{convert -density 150 -units pixelsperinch} seems to work.

\item Concatenating Postscript documents and creating a pdf


\texttt{gs -q -dNOPAUSE -dBATCH -sDEVICE=pdfwrite -sOutputFile=}\texttt{\textit{output-file-name
input-file1 }}\texttt{{[} }\texttt{\textit{input-file2 ...}}\texttt{ ]}~\\
Note that the \texttt{\textit{input-file{*}}} can also be pdf files,
and ps and pdf documents can be mixed in the arguments\texttt{.}

\item Converting pdf documents to postscript: try using gs with -sDEVICE=pswrite.

\item Reluctant documents

\begin{enumerate}
\item You might be able to print reluctant postscript files by converting
them using \texttt{}~\\
\texttt{ps2ps {[}-dLanguageLevel=1] }\texttt{\textit{fromfile tofile}} 
\item Postscript files that are very large can turn into fairly small pdfs if you use \texttt{ps2pdf}.
\item Huge files that are slow to display (render) might be able to be flattened
with \texttt{gimp}. For existing PDF documents that might be very
slow to render some pages (possibly with a huge unflattened figure
on them): 


split up the document with \texttt{gs -dFirstPage=n -dLastPage=m}
\\
\texttt{gimp} the offending pages, saving as postscript\\
join the document back together with \texttt{gs} (see concatenating
above)


\end{enumerate}

\item Badly adjusted page offsets

\begin{enumerate}
\item For source from a \LaTeX\ document: try \texttt{dvips -t letter -f
<dvifile>}
\item Look for the \texttt{align.ps} file in the ghostscript package; there
are instructions in there for adjusting the margins using \texttt{gs}.
Create a \texttt{margin.ps} file containing


{\tt \%\% $<<$ /.HWMargins [ml mb mr mt] /Margins [x y] $>>$ setpagedevice \\
\%\%   ml = L * 72, mb = B * 72, mr = R * 72, mt = T * 72, \\
\%\%   x = (1 - H) * 720.0, y = (V - 1) * 720.0 \\
$<<$ /.HWMargins [0 0 0 0] /Margins [-180 -360] $>>$ setpagedevice} 

with the appropriate margins then add the \texttt{margin.ps} file
to the list of input files. 

\item Consult some of these for the problems of A4 versus Letter size:
\begin{itemize}
\item \url{http://amath.colorado.edu/documentation/LaTeX/reference/faq/a4.html}
\item \url{http://dam.mellis.org/2003/12/a4_vs_letter/}
\item \url{http://mintaka.sdsu.edu/GF/bibliog/latex/LaTeXtoPDF.html}
\end{itemize}

\end{enumerate}

\end{enumerate}

\section*{StarOffice Options}

\begin{tabular}{ll}
\texttt{-minimized}  & keep startup bitmap minimized\tabularnewline
\texttt{-help/-h/-?} &  show the help message and exit\tabularnewline
\texttt{-writer} &  create new text document\tabularnewline
\texttt{-calc } & create new spreadsheet document\tabularnewline
\texttt{-draw } & create new drawing\tabularnewline
\texttt{-impress } & create new presentation\tabularnewline
\texttt{-math} &  create new formula\tabularnewline
\texttt{-global } & create new global document\tabularnewline
\texttt{-web} &  create new HTML document\tabularnewline
 & \tabularnewline
\end{tabular}


\section*{X-server Workarounds {\normalsize }\protect \\
\textmd{\normalsize (Cures for some of the insanity in KDE, gnome,
StarOffice and friends)}}

\begin{tabular}{ll}
\texttt{ssh -X} & \tabularnewline
\texttt{konsolekalendar -{}-help} & \tabularnewline
\texttt{soffice -help/-h/-?}  & don't try \texttt{-{}-{[}option]} !\tabularnewline
\texttt{soffice -minimized } & no splash screen\tabularnewline
 & \tabularnewline
\end{tabular}

\clearpage


\section*{Recording a CD}

{[} Obsolete Comment: Star has a Creative CDRW. Speeds are 4,2,24
(writable, rewritable, read). NOTE: Drive does not like fixating in
dummy mode. The SCSI emulator driver is susceptible to locking up
the CD on this configuration, requiring a power cycle reboot from
time to time. ]


\subsubsection*{Modules required}

sg, sr\_mod, loop


\subsubsection*{Blank a rewritable cd }

\begin{itemize}
\item \texttt{cdrecord $-$v blank=fast dev=0,0}
\end{itemize}
Can blank and burn in the same command.


\subsubsection*{Make a filesystem}

\begin{itemize}
\item \texttt{\# For an ext2 filesystem}
\item \texttt{dd if=/dev/zero of=cdimage; mke2fs cdimage; mount $-$o loop
cdimage /mnt; \textbackslash{}}~\\
\texttt{cp $-$a dir /mnt}
\item \texttt{\# For an iso9660 filesystem}
\item \texttt{mkisofs $-$v $-$R $-$o cdimage dir}
\item \texttt{\# Burn it}
\item \texttt{cdrecord $-$v speed=2 dev=0,0 cdimage}
\end{itemize}
Use \texttt{mkhybrid} for a filesystem which can be read by a Mac.


\subsubsection*{In one go}

To burn the contents of the directory \texttt{dir.} Note the double
$-$ for the \texttt{nice} and the final $-$ for the \texttt{cdrecord. }

\begin{itemize}
\item \texttt{nice $--$18 mkisofs $-$J $-$R $-$r dir | cdrecord $-$v
speed=2 dev=0,0 $-$}

\begin{lyxlist}{00.00.0000}
\item [{\texttt{$-$R}}] Rock Ridge extensions
\item [{\texttt{$-$r}}] global read permissions and root ownership
\item [{\texttt{$-$J}}] Joliet extensions
\end{lyxlist}
\end{itemize}

\subsubsection*{Setting defaults}

The default device and speed can be specified in the file \texttt{/etc/default/cdrecord},
to shorten the above commands, e.g.

\begin{itemize}
\item \texttt{nice $--$18 mkisofs $-$J $-$R $-$r dir | cdrecord $-$v
$-$}
\item \texttt{cat /etc/default/cdrecord} \\
\texttt{CDR\_DEVICE=0,0,0 }~\\
\texttt{CDR\_SPEED=2}
\end{itemize}

\end{document}
